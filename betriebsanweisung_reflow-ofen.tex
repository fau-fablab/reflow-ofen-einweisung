%%%%%%%%%%%%%%%%%%%%%%%%%%%%%%%%%%%%%%%%%%%%%%%%%%%%%%%%%%%%%%%
%%	Betriebseinweisung zum Reflow-Ofen nach Vorlage der BGV  %%
%%	COPYRIGHT: (C) 2012-2015 FAU FabLab and others			 %%
%%	Lizenz: CC-BY-SA 3.0 - sofern möglich					 %%
%%%%%%%%%%%%%%%%%%%%%%%%%%%%%%%%%%%%%%%%%%%%%%%%%%%%%%%%%%%%%%%


\documentclass[fontsize=11pt]{scrartcl}
\pagestyle{empty}
\usepackage{lmodern}
\renewcommand*\familydefault{\sfdefault}
\usepackage[utf8]{inputenc}
\usepackage[T1]{fontenc}
\usepackage[ngerman]{babel}
\usepackage[babel,german=quotes]{csquotes}
\usepackage{graphicx}
\usepackage{xcolor}
\usepackage{geometry}
\usepackage{microtype}
\usepackage{enumitem}
\setitemize{itemsep=-0.05em}
\geometry{a4paper, top=20mm, left=16mm, right=17mm, bottom=15mm, headsep=0mm, footskip=0mm}

\usepackage{tikz}
\makeatletter
\setkomafont{section}{\color{white}
    \bfseries\Large
    \begin{tikzpicture}[overlay]
        \draw[blue,fill=blue] (-1.5cm,-6pt) rectangle (\paperwidth, 13pt)
        (\linewidth,16.4pt);
	\end{tikzpicture}}

\usepackage{background}
\backgroundsetup{
	scale=1,	%% Größe (so lassen)
	angle=0,	%% Ausrichtung (Winkel)
	opacity=1,  %% Deckkraft
	contents={\includegraphics[width=\paperwidth,height=\paperheight]{img/blauer_rand.pdf}}
}

\newcommand{\todo}[1]{\textbf{\color{red}TODO: #1 }}
%\setlength{\itemsep}{-2222mm}
\begin{document}

% blauer Rand

\begin{center}
	\LARGE{Betriebsanweisung \enquote{Reflowofen}}
\end{center}


\section{Anwendungsbereich}
\todo{Dieser Abschnitt beschreibt nicht den Anwendungsbereich}
\begin{center}
	Diese Betriebsanweisung fasst die wichtigsten Gefahren und Regeln zusammen.\\
	Für die Bedienung des Reflowofens ist eine unterschriebene Einweisung nötig. Der Text zur Einweisung enthält weitere Informationen.
\end{center}



\section{Gefahren für Mensch und Maschine}

\begin{itemize}
	\item Verbrennungsgefahr an heißen Teilen. Auch das Gehäuse erhitzt sich stark, besonders oben.
	\item Überhitzung der Platine oder des Ofens mit Rauchentwicklung und Brandgefahr.
	\item gesundheitsschädliche Dämpfe
	\item Gesundheitsschäden durch Hautkontakt mit Lötpaste \todo{Gefahrstoff-Betriebsanw. nötig?}
\end{itemize}
\section{Schutzmaßnahmen und Verhaltensregeln}

%\subsection{Generell}
\begin{itemize}
	\item Arbeite sorgfältig und plane mindestens 2 Stunden Zeit ein, bis alles abgekühlt und verräumt ist!
\item Bei Kontakt mit Lötpaste sofort die Hände gründlich mit Seife waschen.
	\item Ofen vorher auf Mängel und mögliche Probleme kontrollieren, alle brennbaren Gegenstände entnehmen.
	\item Hitzeschutzhandschuhe bereithalten
\item Der Ofen muss frei stehen und es darf nichts darauf abgestellt werden. Auf keinen Fall im Regal oder in der Nähe von brennbaren Gegenständen betreiben. Achte darauf, dass keine Kabel mit dem Gehäuse in Berührung kommen.
	\item \textbf{Der Ofen muss ständig beaufsichtigt werden}, solange er eingesteckt oder noch sehr heiß ist.
\item Bestimmungsgemäße Verwendung:
\begin{itemize}
\setlength{\itemsep}{-3pt}
\item Der Reflow-Ofen ist nur für Lötvorgänge geeignet (z.\,B. von SMD-Bauteilen auf eine Platine)
\item Anderweitige Benutzung ist verboten, insbesondere Essen zubereiten und Kleber aushärten.
\item Der Controller darf nicht für andere Geräte verwendet werden.
\end{itemize}
	\item Der Reflowcontroller darf von Benutzern nicht verstellt werden. Dies ist ausschließlich Betreuern mit ausreichender Kenntnis der Bedienungsanleitung erlaubt. Wurden Änderungen vorgenommen, sind diese nach Benutzung wieder auf die Standardeinstellungen zurückzusetzen.
	\item Während des Lötvorgangs entstehen gesundheitsschädliche Dämpfe, deshalb auf gute Belüftung achten! (Fenster aufmachen!) \todo{Ist das wirklich ausreichend?}
	\item Ofen nach Gebrauch ausstecken und -- nachdem alles abgekühlt ist -- Ofen und Zubehör wieder aufräumen.
\end{itemize}

\section{Verhalten bei Störungen und im Gefahrenfall}
\begin{itemize}
	\item Bei Schäden oder Störungen am Gerät: Gerät vom Stromnetz trennen und Betreuer informieren. Schadensmeldung sichtbar an der Maschine anbringen.
	\item Schäden nur vom Fachmann beseitigen lassen.
\end{itemize}

\todo{Verhalten im Brandfall bzw bei Rauchentwicklung}

\section{Verhalten bei Unfällen - Erste Hilfe}
\begin{itemize}
	\item Gerät abschalten. Betreuer informieren. 
	\item Verletzten betreuen. Gegebenenfalls Rettungsdienst rufen.
\end{itemize}

\end{document} 