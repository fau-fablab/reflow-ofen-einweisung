%%%%%%%%%%%%%%%%%%%%%%%%%%%%%%%%%%%%%%%%%%%%%%%%%%%%%%%%%%%%%%%
%%	Betriebseinweisung zum Reflow-Ofen nach Vorlage der BGV  %%
%%	COPYRIGHT: (C) 2012-2015 FAU FabLab and others			 %%
%%	Lizenz: CC-BY-SA 3.0 - sofern möglich					 %%
%%%%%%%%%%%%%%%%%%%%%%%%%%%%%%%%%%%%%%%%%%%%%%%%%%%%%%%%%%%%%%%


\documentclass[fontsize=11pt]{scrartcl}
\pagestyle{empty}
\usepackage{lmodern}
\renewcommand*\familydefault{\sfdefault}
\usepackage[utf8]{inputenc}
\usepackage[T1]{fontenc}
\usepackage[ngerman]{babel}
\usepackage[babel,german=quotes]{csquotes}
\usepackage{graphicx}
\usepackage{xcolor}
\usepackage{geometry}
\usepackage{microtype}
\usepackage{enumitem}
\setitemize{itemsep=-0.05em}
\geometry{a4paper, top=20mm, left=16mm, right=17mm, bottom=15mm, headsep=0mm, footskip=0mm}

\usepackage{tikz}
\makeatletter
\setkomafont{section}{\color{white}
    \bfseries\Large
    \begin{tikzpicture}[overlay]
        \draw[blue,fill=blue] (-1.5cm,-6pt) rectangle (\paperwidth, 13pt)
        (\linewidth,16.4pt);
	\end{tikzpicture}}

\usepackage{background}
\backgroundsetup{
	scale=1,	%% Größe (so lassen)
	angle=0,	%% Ausrichtung (Winkel)
	opacity=1,  %% Deckkraft
	contents={\includegraphics[width=\paperwidth,height=\paperheight]{img/blauer_rand.pdf}}
}

\newcommand{\todo}[1]{\textbf{\color{red}TODO: #1 }}
%\setlength{\itemsep}{-2222mm}
\begin{document}

% blauer Rand

\begin{center}
	\LARGE{Betriebsanweisung \enquote{Reflowofen}}
\end{center}


\section{Anwendungsbereich}
\begin{center}
\textbf{Verwendung des Reflowofens}\\
	Diese Betriebsanweisung fasst die wichtigsten Gefahrenquellen und Regeln zusammen. 	Für die Bedienung des Reflowofens ist eine unterschriebene Einweisung nötig.
\end{center}



\section{Gefahren für Mensch und Maschine}

\begin{itemize}
	\item Verbrennungsgefahr an heißen Teilen. Auch das Gehäuse erhitzt sich auf der Oberseite stark
	\item Rauch- und Feuergefahr bei Überhitzung der Platine oder des Ofens, z.\,B. wenn sich der Sensor löst
	\item gesundheitsschädliche Dämpfe, besonders durch verdampfendes Flussmittel
	\item Gesundheitsschäden durch Hautkontakt mit Lötpaste
	% keine Gefahrstoffbetriebsanweisung nötig, da keine Einstufung zu Lötpaste durch die BG
\end{itemize}

\section{Schutzmaßnahmen und Verhaltensregeln}

%\subsection{Generell}
\begin{itemize}
% Vorbereitung / allgemeines:
	\item Arbeite sorgfältig und plane mindestens 2 Stunden Zeit ein, bis alles abgekühlt und verräumt ist!
	\item Bei Kontakt mit Lötpaste sofort die Hände gründlich mit Seife waschen
	\item Ofen vorher auf Mängel und mögliche Probleme kontrollieren,
%%%%%%%%%%%%%%%%%%%%%%
% Hitze / Brand:
 alle brennbaren Gegenstände entnehmen
	\item Es dürfen keine brennbaren Gegenstände auf dem Ofen oder in der Nähe abgestellt werden
	\item Hitzeschutzhandschuhe bereithalten
	% TODO bessere Formulierung finden, der Ofen lässt sich sehr gut im Regal betreiben, sofern er sicher steht (Fliese)
	%\item Der Ofen muss frei stehen und es darf nichts darauf abgestellt werden. Auf keinen Fall im Regal oder in der Nähe von brennbaren Gegenständen betreiben. Achte darauf, dass keine Kabel mit dem Gehäuse in Berührung kommen
	\item \textbf{Der Ofen muss ständig beaufsichtigt werden}, solange er eingeschaltet oder noch heiß ist
%%%%%%%%%%%%%%%%%%%%%%
% "Fehlbenutzung"
	\item Bestimmungsgemäße Verwendung: Der Ofen ist zum Reflow-Löten mit Controller vorgesehen.
\begin{itemize}
	\setlength{\itemsep}{-3pt}
	\item Anderweitige Benutzung muss auf jeden Fall vorher mit einem Betreuer abgeklärt werden.
	Es darf keine Brand-, Rauch- und Gestankgefahr bestehen.
	\item \textbf{Lebensmittel sind absolut verboten!}
	\item Bei Verwendung des Ofens ohne Controller müssen Sensor und Sensor-Platine herausgenommen werden, damit sie nicht überhitzen können.
	%\item Der Controller darf nicht für andere Geräte verwendet werden.
\end{itemize}
%	\item Der Reflowcontroller darf von Benutzern nicht verstellt werden. Dies ist ausschließlich Betreuern mit ausreichender Kenntnis der Bedienungsanleitung erlaubt. Wurden Änderungen vorgenommen, sind diese nach Benutzung wieder auf die Standardeinstellungen zurückzusetzen
%%%%%%%%%%%%%%%%%%%%%%
% sonstiges, Nachbereitung:
	\item Während des Lötvorgangs entstehen gesundheitsschädliche Dämpfe, deshalb für vollen Durchzug sorgen, also \textbf{Fenster und Dachfenster aufmachen! Wenn das nicht gewährleistet werden kann, (z.B. im Winter, Starkregen) darf der Ofen nicht verwendet werden.}
	\item Ofen nach Gebrauch ausschalten und Zubehör verräumen
\end{itemize}

\section{Verhalten bei Störungen und im Gefahrenfall}

\begin{itemize}
	\item Bei Schäden oder Störungen am Gerät: Gerät vom Stromnetz trennen und Betreuer informieren. Schadensmeldung sichtbar an der Maschine anbringen
	\item Bei starker Rauchentwicklung oder Feuererscheinung Gerät vom Stromnetz trennen, Platine entfernen, ggf. mit Feuerlöscher ablöschen
	\item Schäden nur vom Fachmann beseitigen lassen
\end{itemize}

\section{Verhalten bei Unfällen - Erste Hilfe}

\begin{itemize}
	\item Gerät abschalten. Betreuer informieren
	\item Verletzten betreuen. Gegebenenfalls Rettungsdienst rufen
\end{itemize}

\end{document}
