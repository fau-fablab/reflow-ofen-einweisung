%%%%%%%%%%%%%%%%%%%%%%%%%%%%%%%%%%%%%%%%%%%%%%%%%%%%%%%%%%%%%%%
%%	Betriebseinweisung zum Reflow-Ofen nach Vorlage der BGV  %%
%%	COPYRIGHT: (C) 2012-2015 FAU FabLab and others			 %%
%%	Lizenz: CC-BY-SA 3.0 - sofern möglich					 %%
%%%%%%%%%%%%%%%%%%%%%%%%%%%%%%%%%%%%%%%%%%%%%%%%%%%%%%%%%%%%%%%


\documentclass[fontsize=9pt]{scrartcl}
\pagestyle{empty}
\usepackage{lmodern}
\renewcommand*\familydefault{\sfdefault}
\usepackage[utf8]{inputenc}
\usepackage[T1]{fontenc}
\usepackage[ngerman]{babel}
\usepackage[babel,german=quotes]{csquotes}
\usepackage{graphicx}
\usepackage{xcolor}
\usepackage{geometry}
\geometry{a4paper, top=15mm, left=20mm, right=20mm, bottom=15mm, headsep=0mm, footskip=0mm}

\usepackage{tikz}
\makeatletter
\setkomafont{section}{\color{white}
    \bfseries\Large
    \begin{tikzpicture}[overlay]
        \draw[blue,fill=blue] (-1.5cm,-6pt) rectangle (\paperwidth, 13pt)
        (\linewidth,16.4pt);
	\end{tikzpicture}}

\usepackage{background}
\backgroundsetup{
	scale=1,	%% Größe (so lassen)
	angle=0,	%% Ausrichtung (Winkel)
	opacity=1,  %% Deckkraft
	contents={\includegraphics[width=\paperwidth,height=\paperheight]{img/blauer_rand.pdf}}
}

\newenvironment{smallitemize}{\begin{itemize}\itemsep -3pt}{\end{itemize}}



\begin{document}

% blauer Rand

\begin{center}
	\LARGE{Betriebsanweisung \enquote{Reflowofen}}
\end{center}


\section{Anwendungsbereich}
\begin{center}
	Diese Betriebsanweisung fasst die wichtigsten Gefahren und Regeln zusammen.\\
	Für die Bedienung des Reflowofen ist eine unterschriebene Einweisung nötig.\\
	Vorbereitende Arbeiten, insbesondere das Auftragen der Lötpaste, wird gesondert behandelt.\\
\end{center}

\section{Gefahren für Mensch und Maschine}

\begin{smallitemize}
	\item Verbrennungsgefahr an heißen Teilen, auch das Gehäuse erhitzt sich stark.
	\item Überhitzung des Ofens und dadurch Brandgefahr.
	\item Hautkontakt mit Lötpaste.
	\end{smallitemize}

\section{Schutzmaßnahmen und Verhaltensregeln}

\begin{itemize}
	\item Generell:
	\begin{smallitemize}
		\item Ofen auf Mängel und mögliche Probleme kontrollieren.
		\item Während der Ofen warm bzw. in Betrieb ist, dürfen keine brennbaren Gegenstände im Ofen sein. 
		\item Der Ofen muss frei stehen und es darf nichts auf dem Ofen abgestellt werden. Auf keinen Fall im Regal oder in der Nähe von brennbaren Gegenständen betreiben.
		\item Der Ofen darf nicht unbeaufsichtigt betrieben werden.
		\item Ausschließlich den Reflow Ofen an den Reflowcontroller anschließen. Niemals andere Geräte.
		\item Maschine nach Gebrauch abschalten und - nachdem alles abgekühlt ist - Ofen und Zubehör wieder aufräumen.
		\item Der Reflowcontroller darf vom Laien nicht verstellt werden. Dies ist ausschließlich Betreuern erlaubt. Detailierte Informationen hierzu befinden sich in der Anleitung zum Gerät. Wurden Änderungen vorgenommen, sind diese nach Benutzung wieder auf die Standardeinstellungen zurückzusetzen.
		\item Bei Hautkontakt mit Lötpaste entsprechende Stelle mit Seife abwaschen. 
		\end{smallitemize}
	\item Vorbereitung:
	\begin{smallitemize}
		\item Das Zubehör, das sich im Ofen befindet, herausnehmen.
		\item Temperatursensor in den Reflowcontroller stecken.
		\item Reflowcontroller direkt in die Steckdose stecken und den Reflowofen in die Schukodose stecken, die am Reflowcontroller angeschlossen ist.
		\item Temperatursensor gewissenhaft mit Kaptonband auf der Platine befestigen. Löst sich der Temperatursensor im Betrieb, muss der Reflowprozess in jedem Fall abgebrochen werden, da es sonst zu einer Überhitzung der Platine kommt.
		\item Platine mit montiertem Temperatursensor auf das Gitter im Ofen legen. Das Gitter sollte sich auf der obersten Schiene befinden.
		\item Timer am Ofen auf maximale Zeit stellen. Läuft der Timer ab, schaltet sich der Ofen ab und der Reflowprozess geht schief.

	\end{smallitemize}
	\item Im Betrieb:
	\begin{smallitemize}
		\item Reflowcontroller am Hauptschalter oben rechts einschalten.
		\item Die Ofentür darf während des gesammten Prozesses nicht geöffnet werden.
		\item Der voreingestellte Reflowprozess wird durch drücken auf \enquote{Start} gestartet.
		\item Während dem des Reflowprozesses ist der Ofen ständig zu beaufsichtigen und darauf zu achten, dass sich der Temperatursensor nicht löst. Falls sich dieser lösen sollte, mit \enquote{Stop} abbrechen. Niemals versuchen, den Sensor im Betrieb wieder zu befestigen, sondern erst vollständig abkühlen lassen.
		\item Aktuelle Temperatur und Reflowphase werden im Display angezeigt.
		\item Nach dem Lötprozess kann die Platine entnommen werden sobald der Ofen ausreichend abgekühlt ist und keine Verbrennungsgefahr mehr besteht.
		\end{smallitemize}
	\item Nach dem Betrieb:
	\begin{smallitemize}
		\item Ofen vollständig abkühlen lassen
		\item Zubehör im Ofen verstauen und den Ofen wieder wegräumen
	\end{smallitemize}

\end{itemize}

\section{Verhalten bei Störungen und im Gefahrenfall}
\begin{smallitemize}
	\item Bei Schäden oder Störungen am Gerät: Gerät vom Stomnetz trennen und Betreuer informieren. Schadensmeldung sichtbar an der Maschine anbringen.
	\item Schäden nur vom Fachmann beseitigen lassen.
\end{smallitemize}

\section{Verhalten bei Unfällen - Erste Hilfe}
\begin{smallitemize}
	\item Gerät abschalten. 
	\item Betreuer informieren. Gegebenenfalls Rettungsdienst rufen.
	\item Verletzten betreuen.
\end{smallitemize}



\end{document} 