%%%%%%%%%%%%%%%%%%%%%%%%%%%%%%%%%%%%%%%%%%%%%%%%
% COPYRIGHT: (C) 2012-2015 FAU FabLab and others
% Bearbeitungen ab 2015-02-20 fallen unter CC-BY-SA 3.0
% Sobald alle Mitautoren zugestimmt haben, steht die komplette Datei unter CC-BY-SA 3.0. Bis dahin ist der Lizenzstatus aller alten Bestandteile ungeklärt.
%%%%%%%%%%%%%%%%%%%%%%%%%%%%%%%%%%%%%%%%%%%%%%%%


\newcommand{\basedir}{fablab-document}
\documentclass{\basedir/fablab-document}

% \usepackage{fancybox} %ovale Boxen für Knöpfe - nicht mehr benötigt
\usepackage{amssymb} % Symbole für Knöpfe
% \usepackage{subfigure,caption}
\usepackage{eurosym}
\usepackage{tabularx} % Tabellen mit bestimmtem Breitenverhältnis der Spalten
\usepackage{wrapfig} % Textumlauf um Bilder
\usepackage{pdfpages}

\renewcommand{\texteuro}{\euro}

% \linespread{1.2}

\author{kontakt@fablab.fau.de}
\title{Einweisung Reflow-Ofen}

\begin{document}

\color{red}
\hrule
\begin{center}
\large{Achtung! Einweisung ist noch in Arbeit!}
\vspace{0.1cm}
\end{center}
\hrule
\color{black}
\tableofcontents
% \KOMAoptions{BCOR=0pt}

\section{Gefahren und Verhaltensregeln}
Gefahren und wichtige Regeln sind in der Betriebsanweisung (blaue Seite) zusammengefasst. Lese diese dir aufmerksam durch; du musst die Regeln auswendig kennen, um sicher mit dem Gerät zu arbeiten!
\includepdf[pages=1,scale=1,offset= 14.5 -8.5]{betriebsanweisung_reflow-ofen.pdf}

\section{Aufbringen der Lötpaste}
Bei Hautkontakt mit Lötpaste entsprechende Stelle mit Seife abwaschen. 

\begin{itemize}
\item Gesamt 2h Zeit haben, da der Ofen nachher auch wieder aufgeräumt werden will!
\item Platine reinigen
\item Schablone fertigen aus Overhead-Folie mit Schneideplotter. \\ Dazu die Plotter-Einweisung beachten, insbesondere darf die Tiefenbegrenzung nicht zu groß eingestellt werden. Nach Benutzung muss wieder auf normal zurückgestellt werden!
\item Paste vorher wiegen (mit Deckel), Wert notieren
\item Rakeldingsi zusammenbauen
\item Paste aufrakeln
\item Paste danach wiegen (mit Deckel), Differenz zu vorher bezahlen.
\item Platine bestücken, Kunststoffbauteile am besten weglassen oder mit Kapton etwas schützen
\end{itemize}

\todo{genauer beschreiben}

\section{Reflow}
\subsection{Vorbereitung des Ofens}
\begin{itemize}
\item Hitzeschutzhandschuhe bereitlegen: In derder Schublade Arbeitsschutz in der Werkbank finden sich gelbe Hitzeschutzhandschuhe für die T-Shirt-Presse, die auch für den Ofen verwendet werden dürfen.
\item Das Zubehör, das sich im Ofen befindet, herausnehmen.
		\item Temperatursensor in den Reflowcontroller stecken.
		\item Reflowcontroller direkt in die Steckdose stecken und den Reflowofen in die Schukodose stecken, die am Reflowcontroller angeschlossen ist.
		\item Temperatursensor gewissenhaft mit Kaptonband auf der Platine befestigen. Löst sich der Temperatursensor im Betrieb, muss der Reflowprozess in jedem Fall abgebrochen werden, da es sonst zu einer Überhitzung der Platine kommt.
		\item Platine mit montiertem Temperatursensor auf das Gitter im Ofen legen. Das Gitter sollte sich auf der obersten Schiene befinden.
		\item Timer am Ofen auf maximale Zeit stellen. Läuft der Timer ab, schaltet sich der Ofen ab und der Reflowprozess geht schief.
\end{itemize}
\subsection{Durchführung}
\begin{itemize}
		\item Reflowcontroller am Hauptschalter oben rechts einschalten.
		\item Die Ofentür darf während des gesammten Prozesses nicht geöffnet werden. \todo{Verhalten bei Störungen / Brand? Man muss die Platine sofort raustun! Hitzeschutzhandschuhe müssen bereitliegen!}
		\item Der voreingestellte Reflowprozess wird durch drücken auf \enquote{Start} gestartet.
		\item Während des Reflowprozesses ist der Ofen ständig zu beaufsichtigen und darauf zu achten, dass sich der Temperatursensor nicht löst. Falls sich dieser lösen sollte, mit \enquote{Stop} abbrechen. Niemals versuchen, den Sensor im Betrieb wieder zu befestigen, sondern erst vollständig abkühlen lassen.
		\item Aktuelle Temperatur und Reflowphase werden im Display angezeigt.
		\item Nach dem Lötprozess kann die Platine entnommen werden sobald der Ofen ausreichend abgekühlt ist und keine Verbrennungsgefahr mehr besteht.
\end{itemize}
\subsection{Abschließend}
\begin{itemize}
	\item Ofen vollständig abkühlen lassen
	\item Zubehör im Ofen verstauen und den Ofen wieder wegräumen
\end{itemize}





\section{Kosten}
Du zahlst im Kassenterminal nur den Preis für die verbrauchte Lötpaste (unbedingt vor und nach der Verwendung mit der Feinwaage wiegen).
\todo{stimmt nicht, plus Plotterkosten!}


\ccLicense{reflow-ofen-einweisung}{Einweisung Reflow Ofen}

\end{document}
