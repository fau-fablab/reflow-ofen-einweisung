%%%%%%%%%%%%%%%%%%%%%%%%%%%%%%%%%%%%%%%%%%%%%%%%
% COPYRIGHT: (C) 2012-2015 FAU FabLab and others
% Bearbeitungen ab 2015-02-20 fallen unter CC-BY-SA 3.0
% Sobald alle Mitautoren zugestimmt haben, steht die komplette Datei unter CC-BY-SA 3.0. Bis dahin ist der Lizenzstatus aller alten Bestandteile ungeklärt.
%%%%%%%%%%%%%%%%%%%%%%%%%%%%%%%%%%%%%%%%%%%%%%%%


\newcommand{\basedir}{fablab-document}
\documentclass{\basedir/fablab-document}
% \usepackage{fancybox} %ovale Boxen für Knöpfe - nicht mehr benötigt
\usepackage{amssymb} % Symbole für Knöpfe
% \usepackage{subfigure,caption}
\usepackage{eurosym}
\usepackage{tabularx} % Tabellen mit bestimmtem Breitenverhältnis der Spalten
\usepackage{wrapfig} % Textumlauf um Bilder
\usepackage{pdfpages}

\renewcommand{\texteuro}{\euro}

% \linespread{1.2}

\author{kontakt@fablab.fau.de}
\title{Einweisung Reflowofen und Lötpaste}

\begin{document}

\tableofcontents

\section{Gefahren und Verhaltensregeln}
Gefahren und wichtige Regeln sind in der Betriebsanweisung (blaue Seite) zusammengefasst. Lese diese dir aufmerksam durch; du musst die Regeln auswendig kennen, um sicher mit dem Gerät zu arbeiten!
% TODO HACK wieso muss hier das offset rein.... geht das nicht hübscher?
\includepdf[pages=1,scale=1,offset= 14.5 -8.5]{betriebsanweisung_reflow-ofen.pdf}

\section{Grundsätzliches zur Lötpaste}
Lötpaste enthält sehr kleine Partikel Lötzinn und geht schwer von den Fingern ab, wenn damit in Berührung gekommen wurde. Daher ist nach jeder Berührung der Paste mit der Haut die entsprechende Stelle zu mit Seife waschen. Zusätzlich ist in der Lötpaste viel mehr und auch ein agressiveres Flussmittel enthalten, als im Lötdraht. Auch dies sollte möglichst nicht mit der Haut in Berührung gebracht werden. Achtung: Lötpaste für Stencil vor Benutzung gründlich umrühren.

\section{Aufbringen der Lötpaste}
Auf jedes SMD-Pad, auf dem nachher ein Bauteil platziert werden soll, muss die richtige Menge Lötpaste aufgebracht werden. Dies geht auf zwei Methoden:
\begin{itemize}
	\item Stencil (Schablone) und Rakel
	\item Dispenser oder Spritze
\end{itemize}
Stencil wird manchmal vom Platinenhersteller mitgeliefert oder kann mit dem Schneideplotter aus Overheadprojektorfolie selbst hergestellt werden. Achtung, hierbei entstehen Kosten, die vom Nutzer zu tragen sind. Für größere Stückzahlen ist diese Vorgehensweise zu empfehlen.\\
Für Einzelstücke geht es meist schneller und einfacher, die Paste mit einer Spritze oder einem Lötpastendispenser aufzutragen, weil hier keine Schablone erstellt und gefertigt werden muss. Im FabLab ist aktuell leider noch kein Dispenser vorhanden.

\subsection{Stencil}
\begin{itemize}
\item Gesamt 2h Zeit haben, da der Ofen nachher auch wieder aufgeräumt werden will!
\item Platine reinigen
\item Schablone fertigen aus Overhead-Folie mit Schneideplotter. \\ Dazu die Plotter-Einweisung beachten, insbesondere darf die Tiefenbegrenzung nicht zu groß eingestellt werden. Nach Benutzung muss wieder auf normal zurückgestellt werden!
\item Paste vorher wiegen (mit Deckel), Wert notieren
\item Rakel zusammenbauen
\item Paste aufrakeln
\item Paste danach wiegen (mit Deckel), Differenz zu vorher bezahlen.
\item Platine bestücken, Kunststoffbauteile sollten entweder laut Datenblatt die Termperatur aushalten oder erst nachher bestückt werden.
\end{itemize}

\subsection{Dispenser}
noch nix da. 
%todo.


\section{Einleitung zum Gerät}
Der Ofen ist ein handelsüblicher Pizzaofen, auch wenn man wegen der Platinenrückstände darin jetzt keine Pizza mehr machen kann. Da die Temperaturregelung des Ofens selber viel zu ungenau ist, wird der Ofen einfach auf dauerhaft an gestellt und ein vorgeschalteter Controller schaltet passend den Ofen an und aus. Der Controller misst die Temperatur genau auf der Oberfläche einer Beispielplatine, die neben die eigentliche Platine gelegt wird. Auf Objekten mit anderer Reflektivität ist die Temperatur vollkommen anders als auf der Messplatine (siehe Physikbuch: Schwarzkörperstrahler und Reflektivität), ebenso an anderen Zonen des Ofens, weshalb sich diese Art der Messung ausschließlich fürs Reflow-Löten eignet.

% TODO noch was sinnvolles zum Temperatursensor schreiben, mechanische Empfindlichkeit und so

\section{Reflow-Prozess}
Der Ofen ist vom Temperaturprofil her für bleihaltige und bleifreie Paste eingestellt und darf nur von Betreuern mit entsprechender Erfahrung verändert werden. Das Profil ist nach Fertigstellung der entsprechenden Arbeit wieder auf den voreingestellten Wert zu ändern!

\subsection{Vorbereitung des Ofens}
Der Ofen steht im Moment hinten am letzten E-Werkstatt-Arbeitsplatz und ist aufgebaut.
\begin{itemize}
	\item Der Ofen ist so aufzustellen, dass er keinen direkten Kontakt zu brennbarem Material haben kann. Besonders die Oberseite wird im Betrieb heiß.
	\item Hitzeschutzhandschuhe bereitlegen: In der Schublade \enquote{ W9 - Arbeitsschutz} unter der Werkbank finden sich gelbe Hitzeschutzhandschuhe für die T-Shirt-Presse, die auch für den Ofen verwendet werden dürfen.
	\item Das Zubehör, das sich im Ofen befindet, herausnehmen.
	\item Temperatursensor in den Reflowcontroller stecken.
	\item Reflowcontroller direkt in die Steckdose stecken und den Reflowofen in die Schukodose stecken, die am Reflowcontroller angeschlossen ist.
	\item Temperatursensor gewissenhaft mit Kaptonband auf der Platine befestigen, sofern er nicht schon auf einer kleinen Platine im Ofen befestigt ist. Diese Platine ist auch immer dann zu verwenden, wenn auf der eigendlich zu Lötenden Platine kein Platz für den Sensor ist. Löst sich der Temperatursensor im Betrieb, muss der Reflowprozess in jedem Fall abgebrochen werden, da es sonst zu einer Überhitzung der Platine kommt.
	\item Platine mit montiertem Temperatursensor auf das Gitter im Ofen legen. Das Gitter sollte sich soweit möglich in der Mitte des Ofens befinden.
	\item Alternativ den auf der kleinen Platine montierten Sensor so neben die eigendliche Platine legen, dass der Sensor nach oben weist.
	\item Timer und Temperaturregler direkt am Ofen auf Maximalwert. Läuft der Timer ab, schaltet sich der Ofen ab und der Reflowprozess geht schief.
\end{itemize}


\subsection{Durchführung}
\begin{itemize}
	\item Reflowcontroller am Hauptschalter oben rechts einschalten.
	\item Der voreingestellte Reflowprozess wird durch drücken auf \enquote{Start} gestartet.
	\item Die Ofentür darf während des gesamten Prozesses nicht geöffnet werden.
	\item Wenn sich im Ofen dichter Rauch bildet, Plastikbauteile schmelzen oder Anzeichen eines Brandes auftreten, sofort den Reflowconroller abschalten, Stecker ziehen und die Platine aus dem Ofen entfernen!
	\item Während des Reflowprozesses ist der Ofen ständig zu beaufsichtigen und darauf zu achten, dass sich der Temperatursensor nicht löst. Falls sich dieser lösen sollte, mit \enquote{Stop} abbrechen. Nicht versuchen, den Sensor im Betrieb wieder zu befestigen, sondern erst vollständig abkühlen lassen.
	\item Temperatur und Reflowphase werden im Display angezeigt.
	\item Nach dem Lötprozess kann die Platine entnommen werden sobald der Ofen ausreichend abgekühlt ist.
\end{itemize}
\subsection{Abschließend}
Auch hier gilt: nur wenn der Ofen vorher abgebaut im Regal stand, abbauen. Sonst nur Reflowcontroller abschalten und Zubehör im Ofen verstauen.

Abbauen wie folgt:
\begin{itemize}
	\item Ofen vollständig abkühlen lassen
	\item Zubehör im Ofen verstauen und den Ofen wieder wegräumen
\end{itemize}



\section{Kosten}
Am Kassenterminal ist die verbrauchte Lötpaste nach Gewcht (in Gramm) zu bezahlen. Dafür muss die Paste unbedingt vor und nach der Verwendung mit der Feinwaage gewogen werden! Wenn eine Folie als Stencil verwendet wurde und diese im FabLab erstellt wurde, ist diese auch als Schneideplotterkosten OHP-Folie zu bezahlen.


\ccLicense{reflow-ofen-einweisung}{Einweisung Reflow Ofen}

\end{document}
